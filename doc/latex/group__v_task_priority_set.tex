\hypertarget{group__v_task_priority_set}{}\section{v\+Task\+Priority\+Set}
\label{group__v_task_priority_set}\index{v\+Task\+Priority\+Set@{v\+Task\+Priority\+Set}}
task. h 
\begin{DoxyPre}void vTaskPrioritySet( TaskHandle\_t xTask, UBaseType\_t uxNewPriority );\end{DoxyPre}


I\+N\+C\+L\+U\+D\+E\+\_\+v\+Task\+Priority\+Set must be defined as 1 for this function to be available. See the configuration section for more information.

Set the priority of any task.

A context switch will occur before the function returns if the priority being set is higher than the currently executing task.


\begin{DoxyParams}{Parameters}
{\em x\+Task} & Handle to the task for which the priority is being set. Passing a N\+U\+LL handle results in the priority of the calling task being set.\\
\hline
{\em ux\+New\+Priority} & The priority to which the task will be set.\\
\hline
\end{DoxyParams}
Example usage\+: 
\begin{DoxyPre}
void vAFunction( void )
\{
TaskHandle\_t xHandle;
\begin{DoxyVerb}// Create a task, storing the handle.
xTaskCreate( vTaskCode, "NAME", STACK_SIZE, NULL, tskIDLE_PRIORITY, &xHandle );

// ...

// Use the handle to raise the priority of the created task.
vTaskPrioritySet( xHandle, tskIDLE_PRIORITY + 1 );

// ...

// Use a NULL handle to raise our priority to the same value.
vTaskPrioritySet( NULL, tskIDLE_PRIORITY + 1 );
\end{DoxyVerb}

\}
  \end{DoxyPre}
 